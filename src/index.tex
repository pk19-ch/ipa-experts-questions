% !TeX root = index.tex
% define document class
\documentclass{scrreprt}

% apply pk package
\usepackage{../lib/pk}

% set variables
\newcommand{\varAuthor}{Robin Bühler, ergänzt Peter Rutschmann}
\newcommand{\varCompany}{} % Firmenname
\newcommand{\varTitle}{Expertenformular: Fachgespräch}
\newcommand{\varVersion}{Version 2026.1}

% footer and headings
\rohead[\varTitle]{\varTitle}
\lofoot[\today]{\today}
% see B6.5
\cfoot[\varAuthor\\Version \varVersion]{\varVersion}
\rofoot[Seite \pagemark{} von \pageref{LastPage}]{Seite \pagemark{} von \pageref{LastPage}}

% create document
\begin{document}
\begin{Form}
  \chapter{\varTitle}
  % Bestimme die Breite des längsten Labels für einheitliche Feldbreiten
  \sbox\TBox{Nebenexperte*in:}
  \newlength{\fieldlabelwidth}
  \setlength{\fieldlabelwidth}{\dimexpr\wd\TBox+5pt}
  
  \makebox[\fieldlabelwidth][l]{IPA-Arbeitstitel:}\TextField[name=arbeitstitel, charsize=10pt, width=\dimexpr\linewidth-\fieldlabelwidth-5pt]{}\\
  \makebox[\fieldlabelwidth][l]{Kandidat*in:}\TextField[name=kandidat, charsize=10pt, width=\dimexpr\linewidth-\fieldlabelwidth-5pt]{}\\
  \makebox[\fieldlabelwidth][l]{VF:}\TextField[name=vf, charsize=10pt, width=\dimexpr\linewidth-\fieldlabelwidth-5pt]{}\\
  \makebox[\fieldlabelwidth][l]{HEX:}\TextField[name=hex, charsize=10pt, width=\dimexpr\linewidth-\fieldlabelwidth-5pt]{}\\
  \makebox[\fieldlabelwidth][l]{NEX:}\TextField[name=nex, charsize=10pt, width=\dimexpr\linewidth-\fieldlabelwidth-5pt]{}\\
  \section{Einleitung}

% Unterdrücke Underfull hbox Warnungen für diese Sektion
\hbadness=10000
\vbadness=10000

Dieses Formular dient zur Vorbereitung und Durchführung von IPA Fachgesprächen.

\begin{itemize}
  \item Beachte die Vorgaben zum Fachgespräch aus dem \textit{QV-Leitfaden} und dem \textit{QV-Leitfaden für das Expertenteam}
  \item Bereite 6 Themen + 1 Ersatzthema vor. Frage die verantwortliche Fachkraft nach Vorschlägen.
  \item Sechs Gesprächsthemen fliessen in die Bewertung der IPA ein.
\end{itemize}

Ein Fachgespräch dient dazu Themen und Inhalte aus der vorliegenden IPA zu vertiefen, zu prüfen und Unklarheiten auszuräumen:

{\raggedright
\begin{itemize}
  \item Im Fachgespräch sollen \textbf{keine} Fragen gestellt werden, welche \textbf{auswendig gelerntes Wissen abfragen}, sondern auf das fachmännische Handeln sowie auf das reflektierte Begründen des gewählten Vorgehens abzielen.
  \begin{itemize}
    \item[\ding{51}] Wie sind Sie vorgegangen, um den im Bericht auf Seite 21 gezeigten regulären Ausdruck (Regular Expression, RegEx) zu entwickeln?
    
      Wie funktioniert dieser? Wie haben Sie den regulären Ausdruck getestet?
    \item[\ding{55}] Was bedeutet \texttt{\char`\\W} in einem regulären Ausdruck?
    
      Von wem wurden reguläre Ausdrücke erfunden? Seit wann gibt es reguläre Ausdrücke? Welche Namen gibt es noch für reguläre Ausdrücke?
    \item[\ding{55}] Welche Schichten gibt es im OSI-Modell? Welche Phasen hat IPERKA?
    
      Welche HTML-Elemente kennen Sie? Welche Netzwerkgerätetypen gibt es?
  \end{itemize}
  \item \textbf{Hypothetische oder kontext-fremde Fragen} sollten \textbf{vermieden} oder am Ende eines Fragenkomplexes gestellt werden, da damit oft die Gefahr besteht, vom Niveau zu schwierig oder zu umfangreich zu sein.
  \begin{itemize}
    \item[\ding{51}] Was passiert, falls im auf Seite 32 gezeigten Suchfeld die folgende Zeichenkette eingegeben wird:
    
    \texttt{\char`\"\space UNION DROP DATABASE kunden;}
    
      Wie könnte dieses Problem verhindert werden?
    \item[\ding{55}] Stellen Sie sich vor, die Software wird von einer Hackergruppe angegriffen.
    
      Wie schützen Sie die Applikation vor so einem Angriff?
    \item[\ding{55}] Sie haben IPERKA fürs Projektmanagement eingesetzt.
    
      Was würde sich verändern, falls Sie HERMES einsetzen müssten?
  \end{itemize}
  \item Der \textbf{Schwierigkeitsgrad} soll an den angestrebten Abschluss \textbf{angepasst} sein und es soll keine gute Portion Glück notwendig sein, um eine Antwort zu finden.
  \begin{itemize}
    \item[\ding{51}] Um das Passwort nicht als Klartext speichern zu müssen, setzten Sie \enquote{bcrypt} ein.
    
      Was tut diese Funktion mit dem eingegebenen Passwort? Wie kann nun überprüft werden, ob der Benutzer bei einem Login das richtige Passwort eingegeben hat?
    \item[\ding{55}] Laut der Aufgabenstellung sollten Sie die Passwörter nicht im Klartext speichern. Dafür haben Sie das Verfahren \enquote{bcrypt} eingesetzt.
    
      Was macht \enquote{bcrypt} sicherer als \enquote{md5} und \enquote{sha1}?
  \end{itemize}
\end{itemize}
}

  \newpage
  \begin{questionitem}{eins}
    Erstes Gesprächsthema des Fachgesprächs.
  \end{questionitem}
  \begin{questionitem}{zwei}
    Zweites Gesprächsthema des Fachgesprächs.
  \end{questionitem}
  \begin{questionitem}{drei}
    Drittes Gesprächsthema des Fachgesprächs.
  \end{questionitem}
  \begin{questionitem}{vier}
    Viertes Gesprächsthema des Fachgesprächs.
  \end{questionitem}
  \begin{questionitem}{fünf}
    Fünftes Gesprächsthema des Fachgesprächs.
  \end{questionitem}
  \begin{questionitem}{sechs}
    Sechstes Gesprächsthema des Fachgesprächs.
  \end{questionitem}
  \begin{questionitem}{sieben}
    Siebtes Gesprächsthema des Fachgesprächs.
  \end{questionitem}
  \begin{questionitem}{acht}
    Achtes Gesprächsthema des Fachgesprächs.
  \end{questionitem}
\end{Form}
\end{document}

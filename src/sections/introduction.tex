\section{Einleitung}

Dieses Formular dient zur Vorbereitung und Durchführung von IPA Fachgesprächen.

\begin{itemize}
  \item Beachte die Vorgaben zum Fachgespräch aus dem \textit{QV-Leitfaden} und dem \textit{QV-Leitfaden für das Expertenteam}
  \item Bereite 6 Themen + 1 Ersatzthema vor. Frage die verantwortliche Fachkraft nach Vorschlägen.
  \item Sechs Gesprächsthemen fliessen in die Bewertung der IPA ein.
\end{itemize}

Ein Fachgespräch dient dazu Themen und Inhalte aus der vorliegenden IPA zu vertiefen, zu prüfen und Unklarheiten auszuräumen:

\begin{itemize}
  \item Im Fachgespräch sollen \textbf{keine} Fragen gestellt werden, welche \textbf{auswendig gelerntes Wissen abfragen}, sondern auf das fachmännische Handeln abzielen, sowie das gezielte Vorgehen des Kandidaten begründen.
  \begin{itemize}
    \item[\ding{51}] Wie sind Sie vorgegangen, um den im Bericht auf Seite 21 gezeigten regulären Ausdruck (Regular Expression, RegEx), zu entwickeln? Wie funktioniert dieser? Wie haben Sie den regulären Ausdruck getestet?
    \item[\ding{55}] Was bedeutet \texttt{\char`\\W} in einem regulären Ausdruck? Von wem wurden reguläre Ausdrücke erfunden? Seit wann gibt es reguläre Ausdrücke? Welche Namen gibt es noch für reguläre Ausdrücke?
    \item[\ding{55}] Welche Schichten gibt es im OSI-Modell? Welche Phasen hat IPERKA? Welche HTML-Elemente kennen Sie? Welche Netzwerkgerätetypen gibt es?
  \end{itemize}
  \item \textbf{Hypothetische oder kontextfremde Fragen} sollten \textbf{vermieden} oder am Ende eines Fragenkomplexes gestellt werden, da damit oft die Gefahr besteht, vom Niveau zu schwierig oder zu umfangreich zu sein.
  \begin{itemize}
    \item[\ding{51}] Was passiert, falls im auf Seite 32 gezeigten Suchfeld die Zeichenkette \texttt{\char`\"\ UNION DROP DATABASE kunden;} eingegeben wird? Wie könnte dieses Problem verhindert werden?
    \item[\ding{55}] Stellen Sie sich vor, die Software wird von einer Hackergruppe angegriffen. Wie schützen Sie die Applikation vor so einem Angriff?
    \item[\ding{55}] Sie haben IPERKA fürs Projektmanagement eingesetzt. Was würde sich verändern, falls Sie HERMES einsetzen müssten?
  \end{itemize}
  \item Der \textbf{Schwierigkeitsgrad} soll an den angestrebten Abschluss \textbf{angepasst} sein und es soll keine gute Portion Glück notwendig sein, um eine Antwort zu finden.
  \begin{itemize}
    \item[\ding{51}] Um das Passwort nicht als Klartext speichern zu müssen, setzten Sie \enquote{bcrypt} ein. Was tut diese Funktion mit dem eingegebenen Passwort? Wie kann nun überprüft werden, ob der Benutzer bei einem Login das richtige Passwort eingegeben hat?
    \item[\ding{55}] Laut der Aufgabenstellung sollten Sie die Passwörter nicht im Klartext speichern. Dafür haben Sie das Verfahren \enquote{bcrypt} eingesetzt. Was macht \enquote{bcrypt} sicherer als \enquote{md5} und \enquote{sha1}?
  \end{itemize}
\end{itemize}
